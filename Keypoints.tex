\documentclass[10.5pt]{jsarticle}
\usepackage{otf} %Windows上ではコメントアウトしてね
\usepackage{amsmath, amssymb}
\usepackage[dvipdfmx]{graphics,xcolor}
\usepackage{tikz}
\usepackage[framemethod=tikz]{mdframed}
\usepackage{type1cm}
\usepackage{graphicx}
\usepackage{float}
\usepackage{here}
\usepackage{fancyhdr}
\usepackage{lscape} %ページ全体を用いた横向き画像使用時に使用
\usepackage{tabularx}

%マージン設定
%横
\setlength{\textwidth}{165truemm}
\setlength{\hoffset}{-1truein}
\setlength{\oddsidemargin}{25truemm}
%縦
\setlength{\textheight}{257truemm}
\setlength{\voffset}{-1truein}
\setlength{\topmargin}{10truemm}
\setlength{\headheight}{5truemm}
\setlength{\headsep}{5truemm}

%ページ番号設定
\pagestyle{fancy}
\fancyhead[RE]{}
\fancyhead[RO]{\thepage}
\fancyhead[LE]{\thepage}
\fancyhead[LO]{}
\cfoot{}
\renewcommand{\headrulewidth}{0pt}

%図表番号設定
%「図4.4.1」みたいにしたかったらsectionをsubsectionに変える
\makeatletter
\renewcommand{\figurename}{図}
\renewcommand{\thefigure}{\thesection.\arabic{figure}}
\@addtoreset{figure}{section}
\makeatother
\makeatletter
\renewcommand{\tablename}{表}
\renewcommand{\thetable}{\thesection.\arabic{table}}
\@addtoreset{table}{section}
\makeatother

%箇条書き表示設定
\renewcommand{\labelenumi}{(\arabic{enumi})}%第1階層
\renewcommand{\labelenumii}{(\roman{enumii})}%第2階層

%タイトル設定
\title{「通信ネットワーク」 キーポイント}
\author{}
\date{\vspace{-10mm}3年 後期}

%「jlisting.sty」が必要
\usepackage{listings,jlisting}
\def\lstlistingname{リスト}	%キャプションの設定
\lstset{%
language={Java},
basicstyle={\small},%
identifierstyle={\small},%
commentstyle={\small\itshape},%
keywordstyle={\small\bfseries},%
ndkeywordstyle={\small},%
stringstyle={\small\ttfamily},
frame={tb},
breaklines=true,
columns=[l]{fullflexible},%
numbers=left,%
xrightmargin=0zw,%
xleftmargin=3zw,%
numberstyle={\scriptsize},%
stepnumber=1,
numbersep=1zw,%
lineskip=-0.5ex%
}

\begin{document}

%タイトルの出力
\maketitle
\thispagestyle{fancy}

\section{ネットワークを構成する2つの基本要素}
ネットワークは,端末および中心の処理系を示すノードとノード間の通信線路を示すリンクによって構成される.ネットワーク図においてはノードは円で描かれ,リンクはそれらを直線で結ぶように描かれる.

\section{LAN,MAN,WANの正式名称と適用領域}
\begin{enumerate}
	\item{LAN : Local-Area Network}\\
		施設内などに設けられた広がりが数百メートル以内のネットワーク.
	\item{MAN : Metropolitan-Area Network}\\
		都市内の通信に用いられる広がりが数十キロメートル以内のネットワーク.
	\item{WAN : Wide-Area Network}\\
		都市間およびそれ以上の長距離を結ぶネットワーク.広がりは数十キロメートル以上.
\end{enumerate}

\section{ネットワークの各種トポロジー}
\begin{enumerate}
\item{メッシュ : Mesh}\\
ノード同士が規則なく相互に通信しあうネットワーク.全てノードが相互に通信しあう場合を特にフルメッシュ(Full-Mesh)という.
通信には無数の経路が考えられるため障害耐性は高いが,線路が多く敷設費用が高い.
\item{スター : Star}\\
1つのノードを中心ノードとし,そのノードからその他のノードへ放射状にリンクが伸びているネットワーク.
中心ノードは基地局と呼ばれる.
管理のしやすさと費用の面から,一般的な加入者線はこの形態をとる.
ただし中心ノードで障害が発生した場合に全てのネットワークがダウンするため,障害耐性は低い.
\item{リング : Ring}\\
リンクが円を描くように隣接するノード間にのみ存在するネットワーク.
1つの線路で障害が発生した場合にも逆方向から通信が可能であるが,2箇所以上障害が発生した場合通信が一切できなくなる.
\item{バス : Bus}\\
中心にバス線があり,全てのノードはそのバス線と直接リンクで結ばれている.
送信した信号が全ての端末で受信される.
全てのノードがバス線を共有するため,信号の衝突による干渉が発生しないような工夫が必要になる.
\item{トリー : Tree}\\
ルートノードから枝分かれする様に伸びていくネットワーク.
CATVのシステムやUSBのような1対多通信が主になる場合に使われる.
\end{enumerate}
以上のネットワークを図示すると,図\ref{topology}のようになる.
\begin{figure}[h]
	\centering
	\begin{minipage}{0.25\hsize}
		\centering
		\includegraphics[width=0.8\hsize]{files/topology_mesh.ai}\\
		(a)メッシュ
	\end{minipage}
	\begin{minipage}{0.25\hsize}
		\centering
		\includegraphics[width=0.8\hsize]{files/topology_star.ai}\\
		(b)スター
	\end{minipage}
	\begin{minipage}{0.25\hsize}
		\centering
		\includegraphics[width=0.8\hsize]{files/topology_ring.ai}\\
		(c)リング
	\end{minipage}
	\begin{minipage}{0.35\hsize}
		\centering
		\includegraphics[width=0.8\hsize]{files/topology_bus.ai}\\
		(d)バス
	\end{minipage}
	\begin{minipage}{0.35\hsize}
		\centering
		\includegraphics[width=0.8\hsize]{files/topology_tree.ai}\\
		(e)トリー
	\end{minipage}
	\caption{各種トポロジ}\label{topology}
\end{figure}

\section{メッシュ網,スター網,リング網におけるノード数と伝送路数の関係}
全てのノード数を$n$とする.
\begin{enumerate}
\item{メッシュ網}\\
ここではフルメッシュについて考えるものとする.
フルメッシュである場合,全てのノード同士が1つのリンクによって結ばれるため,伝送路数は全てのノードから2つのノードを取り出す組み合わせに等しく,\underline{${}_nC_2$}である.
\item{スター網}\\
中心ノードを除く$n-1$個のノードが各々1つずつ中心ノードとのリンクを持つ.
よって伝送路数は\underline{$n-1$}である.
\item{リング網}\\
全てのノードを頂点として多角形を描くようにリンクが存在するため,伝送路数は$n$角形の辺の数に等しく,\underline{$n$}である.
\end{enumerate}

\section{経路切替え型ネットワークと媒体共有型ネットワークの相違点および具体例}


\section{電話ネットワークにおいて2線式および4線式通信路の違いと使い分けられている理由}


\section{呼量[単位:アーラン(Erlang)]の算出法}
呼量とは,ネットワークを構成する設備の中を流れる信号の量(トラヒック量)を呼量と呼ぶ.
呼量は,一定時間$T$において$n$回生起した呼の保留時間(通話時間)をそれぞれ,$t_1, t_2, \cdots , t_n$とすると,
\[\displaystyle a = \frac{\displaystyle \sum_{i=1}^n t_i}{T}\]
と定義される.

\section{回線交換方式とパケット交換方式の原理および前者に比べた時の後者の長所・短所}


\section{データ通信ネットワークにおける集中型と分散型の違いと特徴}



\section{OSI参照モデルの各層の名称}
OSI参照モデルを図\ref{osi_refmodel}に示す.名称は
\begin{figure}[h]
	\centering
	\includegraphics[width=7.5cm]{files/protocols_osi_model.ai}
	\caption{OSI参照モデル}\label{osi_refmodel}
\end{figure}



\section{媒体共有型ネットワークを分類したときの3つの形式とそれぞれの特徴}
\begin{table}[H]
	\centering
	\caption{媒体共有型ネットワークの分類} \label{media-sharing-network}
	\begin{tabularx}{0.9\hsize}{|l|X|} 
		\hline
		\multicolumn{1}{|c|}{\textbf{種類}}&\multicolumn{1}{c|}{\textbf{概要}}\\\hline\hline
		ランダムアクセス型&各端末が,他端末との調整をすることなくそれぞれの判断で信号を送出する.
							できる限り衝突を減らすため,送出の前と後に伝送媒体をモニターする必要がある.\\\hline
		トークンパッシング型&端末間で信号送出権を巡回させ,できる限り公平に信号を送出できるようにする.
							そのために特殊な制御信号(トークン)を用い,これを受信した端末が信号を送出する.\\\hline
		ポーリンング型&端末に信号送出権を与える役割を担う制御局をネットワーク内に設ける.
							制御曲は送出したい信号を保持している端末に順次送出権を割り当てる.\\\hline
	\end{tabularx}
\end{table}


\section{ランダムアクセス型のプロトコルであるALOHAにおいて,トラヒックとスループットをそれぞれ横軸と縦軸で表したときに得られるカーブの形(略図)そのような形になる定性的理由}
ALOHAにおいて,単位時間あたりの平均$\lambda$回発生するパケットが時間$t$の間に$k$回発生する確率はポアソン分布に従う.
各フレームの持続時間を$T$とすると,全フレームの占める延べ時間は$\lambda T$である.
このときトラヒックは$g=\lambda T$と定義されるから,$\lambda = g/T$と表され,ポアソン分布は次式で与えられることとなる.
\[P(t, k) = \frac{(gt/T)^k}{k!}e^{-gt/T}\]

パケットが有効であることは時間幅$2T$の間に他のパケットが送信されないことであるから,パケットが有効である確率$P$は$t=2T, k=0$として次式で表される.
\[P = P(2T, 0) = e^{-2g}\]
このときスループット$s$はトラヒック$g$と確率$P$の積で与えられる.
\[s = g \times P = ge^{-2g}\]
これをグラフ上に描くと,図\ref{}のように$(g, s) = (0.5, 0.18)$の点で極大値をとることとなる.

\section{ランダムアクセス型のプロトコルであるCSMAの送信アルゴリズムがALOHAから改良されている点およびその効果}
ALOHAは,送出したパケットが正しく受信されなかったことを検出した後,ランダムに時間を空けてもう一度信号を送出する.
これに対し,CSMA(Carrier Sense Multiple Acces)はパケットを送出する前にネットワークの状況を把握(キャリアセンス)し,
他のパケットが存在したら送出を取りやめる.
他のパケットを検出したときには,ランダムに時間を空けてパケットを送出するように試みる.

\section{CSMA/CDの送信制御アルゴリズム}



\section{MACアドレスとは}



\section{FDDIにおける信号制御用フレームの名称と信号制御アルゴリズム}



\section{FDDIでのネットワーク障害復旧のための手法(略図)}



\section{ATM方式でのセルおよびその中のペイロード部分のサイズ}



\section{ATMセルのペイロード長を国際標準化で決める際2つの分野から出された相反する要求}



\section{符号誤り制御における垂直・水平パリティ方式の原理.垂直パリティ方式と比べたときの改良点}



\section{CRC符号の求め方}



\section{ハミング距離と符号誤り検出,符号誤り訂正の関係}



\section{畳み込み符号とビタビ復号アルゴリズムの概要}



\section{TCP/IP階層モデルとOSI参照モデルの対応関係}



\section{TCPおよびIPの正式名称とそれぞれの役割}



\section{IPアドレスおよびサブネットマスク}



\section{IPv6のねらい}



\section{(サブ)ネットワークアドレスとブルードキャストアドレス}



\section{グローバルIPアドレスとプライベートIPアドレスの意味と,それらが使い分けられている理由}



\section{ルーティングテーブルを用いたルータの動作}



\section{ARPおよびDNSの役割}



\section{ドメイン名,IPアドレス,MACアドレスについて,それぞれの意味・目的および相互の変換手法}



\section{コネクション型通信とコネクションレス通信の違い}


\section{IP電話サーバの役割}


\section{携帯電話システムにおけるホームメモリの役割}


\section{携帯電話システムにおけるセルサイズの大小と利点・問題点}


\section{ハンドオーバとは}


\section{ISM帯とは}


\section{無線LANにおけるインフラモードとアドホックモードの違い}


\section{隠れ端末問題およびさらされ端末問題とは(略図)}


\section{隠れ端末問題およびさらされ端末問題を解決するための手法}


\section{リアクティブ型ルーティングの手法(略図)}


\end{document}
